\documentclass[runningheads]{llncs}
\usepackage[utf8]{inputenc}
\usepackage[T1]{fontenc}
\usepackage{graphicx} 
\usepackage{listings}
\usepackage{microtype}
\usepackage{booktabs}
\usepackage{stmaryrd}
\usepackage{mathtools}
\usepackage{amsmath,amssymb}
\usepackage{xcolor}
\usepackage{xspace}
\usepackage{tabto}
\usepackage{wrapfig}
\usepackage{comment}

\usepackage[colorlinks,allcolors=blue]{hyperref}
\usepackage[capitalise]{cleveref}

% Springer-compliant names
\Crefname{equation}{Eq.}{Eqs.}
\Crefname{figure}{Fig.}{Figs.}
\Crefname{tabular}{Table}{Tabs.}
\Crefname{example}{Ex.}{Exs.}
\Crefname{section}{Sec.}{Sects.}
\Crefname{corollary}{Cor.}{Cors.}
\Crefname{proposition}{Prop.}{Props.}
\Crefname{theorem}{Thm.}{Thms.}
\Crefname{lemma}{Lem.}{Lems.}
\Crefname{algorithm}{Alg.}{Algs.}
\Crefname{definition}{Def.}{Defs.}

\definecolor{darkgreen}{rgb}{0.0, 0.53, 0.0}
\definecolor{darkblue}{rgb}{0.21,0.21,0.50}
\definecolor{darkgrey}{rgb}{0.17,0.17,0.17}

\newcommand{\code}[1]{\texttt{#1}}

\def\p#1{\mathrel{\ooalign{\hfil$\mapstochar\mkern 5mu$\hfil\cr$#1$}}}
\newcommand{\pto}{\p\to}

\newcommand{\alt}{\kern5pt\mathrel{\mid}\kern5pt}

\newcommand{\Bool}{\mathbb{B}}
\newcommand{\Nat}{\mathbb{N}}
\newcommand{\Int}{\mathbb{Z}}
\newcommand{\Real}{\mathbb{R}}

\newcommand{\Sort}[2]{#1\langle#2\rangle}

\newcommand{\Seq}[1]{\mathit{Seq}\langle#1\rangle}
\newcommand{\Set}[1]{\mathit{Set}\langle#1\rangle}
\newcommand{\MultiSet}[1]{\mathit{MultiSet}\langle#1\rangle}

\newcommand{\Map}[2]{\mathit{Map}\langle#1, #2\rangle}
\newcommand{\Array}[2]{\mathit{Array}\langle#1, #2\rangle}

\newcommand{\nonterminal}[1]{{\color{darkgreen} \langle \textit{#1} \rangle}}
\newcommand{\terminal}[1]{{\color{darkblue} \code{\bfseries #1}}}

\newcommand{\repn}[1]{{\color{darkgrey}^{#1}}}
\newcommand{\rplus}{{\color{darkgrey}^{+}}}
\newcommand{\rstar}{{\color{darkgrey}^{*}}}
\newcommand{\ropt}{{\color{darkgrey}^{?}}}
\newcommand{\rmid}{\mathrel{\color{darkgrey}\mid}}

\newcommand{\command}[1]{{\color{darkblue} \code{\bfseries #1}}}
\newcommand{\identifier}[1]{{\color{lightblue} \code{#1}}}

\newcommand{\sem}[1]{\llbracket #1 \rrbracket}

\newcommand{\old}{\mathbf{old}}
\newcommand{\free}{\mathrm{free}}

\newcommand{\LP}{\terminal{(}}
\newcommand{\RP}{\terminal{)}}



\author{
    Gidon Ernst \inst{1}
    \and
    Alberto Griggio \inst{2}
    \and
    Martin Jonáš \inst{3}}

\institute{
    LMU Munich, Munich, Germany
    \email{gidon.ernst@lmu.de}
    \and
    Fondazione Bruno Kesseler, Trento, Italy
    \email{griggio@fbk.eu}
    \and
    Masaryk University, Brno, Czech Republic
    \email{martin.jonas@mail.muni.cz}
}

\begin{document}

\maketitle

\begin{abstract}
TODO

\keywords{Intermediate verification language}
\end{abstract}

\section{Introduction}

\clearpage

\section{Syntax}

\begin{align}
\nonterminal{statement}
    & \Coloneqq
        \LP \terminal{sequence}~
            \nonterminal{statement} \rplus
        \RP \tag{B} \\
    & \alt
        \LP \terminal{!}~
            \nonterminal{statement}~
            \nonterminal{attribute}\rplus
        \RP \tag{B} \\[4pt]
    & \alt
        \LP \terminal{assume}~
            \nonterminal{term}
        \RP \tag{AB} \\
    & \alt
        \LP \terminal{assign}~
            \LP \LP \nonterminal{symbol}~\nonterminal{term} \RP\rplus \RP
        \RP \tag{AB} \\[4pt]
    & \alt
        \LP \terminal{call}~\nonterminal{symbol}~
            \LP \nonterminal{term}\rstar \RP~
            \LP \nonterminal{symbol}\rstar \RP
        \RP \tag{P} \\
    & \alt
        \LP \terminal{return} \RP
       \tag{P$^+$} \\[4pt] 
    & \alt
        \LP \terminal{label}~
            \nonterminal{symbol}
        \RP \tag{AU} \\
    & \alt
        \LP \terminal{goto}~
            \nonterminal{symbol}
        \RP \tag{AU} \\[4pt]
    & \alt
        \LP \terminal{if}~
            \nonterminal{term}~
            \nonterminal{statement}~
            \nonterminal{statement}\ropt
        \RP \tag{S} \\
    & \alt
        \LP \terminal{while}~
            \nonterminal{term}~
            \nonterminal{statement}
        \RP \tag{S} \\
    & \alt
        \LP \terminal{break} \RP \alt \LP \terminal{continue} \RP
        \tag{S$^+$} \\[4pt]
    & \alt
        \LP \terminal{havoc}~
            \LP \nonterminal{symbol}\rplus \RP
        \RP \tag{AN} \\
    & \alt
        \LP \terminal{choice}~
            \LP \nonterminal{statement}\rplus \RP
        \RP \tag{SN} \\[8pt]
%
\nonterminal{command}
    & \Coloneqq
        \LP \terminal{declare-var}~
            \nonterminal{symbol}~\nonterminal{sort}
        \RP \notag \\
    & \alt
        \LP \terminal{define-proc}~
            \nonterminal{symbol}~
            \LP \LP \nonterminal{symbol}~\nonterminal{sort}\RP\rstar \RP \notag \\
    & \hspace*{4.42cm} \LP \LP \nonterminal{symbol}~\nonterminal{sort}\RP\rstar \RP \notag \\
    & \hspace*{4.42cm} \LP \LP \nonterminal{symbol}~\nonterminal{sort}\RP\rstar \RP
        \notag \\
    & \hspace*{3.0cm} \nonterminal{statement} \notag \RP \\
    & \alt
        \LP \terminal{annotate-tag}~
            \nonterminal{symbol}~
            \nonterminal{attribute}\rplus
        \RP \notag \\
    & \alt
        \LP \terminal{verify-call}~
            \nonterminal{symbol}~
            \LP \nonterminal{term}\rstar \RP
        \RP \notag \\
    & \alt
        \LP \nonterminal{tool-specific}\terminal{-call}~
            \nonterminal{symbol}~
            \LP \nonterminal{term}\rstar \RP
        \RP \notag \\
    & \alt
        \LP \terminal{get-proof} \RP \alt \LP \terminal{get-counterexample} \RP \notag
\end{align}

\noindent
Atomic commands (A) are those whose evaluation is to be interpreted
as taking one atomic step.
We have the following fragments, but in contrast to SMT-LIB's \terminal{set-logic},
we choose not to declare them upfront
\begin{itemize}
\item The base language (B) includes sequential composition,
      the annotation mechanism,
      as well as two atomic commands for
      parallel assignments and assumptions.
\item The interprocedural fragment (P) includes procedure calls,
      with a possible extension of return statements.
\item The fragment with unstructured control flow (U)
      contains labels and gotos.
\item The fragment with structured control flow (S)
      contains conditionals and while loops,
      with a possible extentions of breaks and continue statements.
      It makes a lot of sense to mandate that whenever
      we know a while loop resp. if statement is translated into labels and gotos,
      then the loop should also have a \terminal{:named} annotation
      that becomes the label of the loop head.
\item The nondeterministic fragment (N)
      includes atomic nondeterministic assignments,
      as well as structured choice.
\item The concurrent fragment (C) is yet to be defined
      and left as an exercise to the reader, respectively,
      for future work.
\item Similarly, we aim to standardize exceptions (E) and a memory interface for the heap (H).
\item It is debatable whether there should be a specification statement
      which introduces a nondeterministic transition as a combination
      of \terminal{havoc} and \terminal{assume}.
      It possibly avoids introduction of additional local variables,
      but introduces \terminal{old} into the expression language of programs,
      which would otherwise only occur in specifications.
\end{itemize}

\noindent
The command language for all fragments encompasses
\begin{itemize}
\item Declarations of global variables,
      which are implicitly in scope in all subsequent procedure declarations.
\item Named procedure declarations,
      with lists of inputs, outputs, and local variables,
      followed by a single body statement.
\item The annotation command permits to link attributes
      to statements with a given tag.
\item The \terminal{verify-call} command takes a similar role as the \terminal{check-sat} command,
      where the given symbol is a procedure name to enter,
      and the list of terms for inputs,
      which may refer to globally declared constants and functions,
      and which may be constrained by prior \terminal{assert} commands.

      The command returns either \terminal{correct} (all specified properties hold),
      \terminal{incorrect} (at least one property is violated),
      or \terminal{unkown} (inconclusive, e.g. due to timeout).
\item Analogously to SMT-LIB, the commands \terminal{get-proof}
      and \terminal{get-counterexample} produce evidence
      for the most-recent result of \terminal{verify-call}.
\item Extentsions on queries supported by tools may introduce additional commands.
\end{itemize}

\section{Properties}

\begin{align}
& \terminal{:assert}
    && \nonterminal{relational-term}
    \tag{GB} \\
& \terminal{:live}
    \tag{FB} \\
& \terminal{:non-live}
    \tag{FB} \\[4pt]
& \terminal{:ghost}
    && \LP \nonterminal{symbol}\rplus \RP
    \tag{P} \\
& \terminal{:ghost}
    && \LP \nonterminal{term}\rplus \RP
    \tag{P} \\
& \terminal{:requires}
    && \nonterminal{term}
    \tag{GP} \\
& \terminal{:ensures}
    && \nonterminal{reltional-term}
    \tag{GP} \\
& \terminal{:invariant}
    && \nonterminal{relational-term}
    \tag{GS} \\
& \terminal{:decreases}
    && \terminal{term}
    \tag{FS} \\
& \terminal{:decreases}
    && \LP \terminal{term}\rplus \RP
    \tag{FS$^+$} \\
& \terminal{:modifies}
    && \LP \nonterminal{symbol}\rplus \RP
    \tag{GPS} \\[4pt]
& \terminal{:ltl}
    && \nonterminal{ltl-formula}
    \tag{LTL}
\end{align}

\noindent
Properties are specified as annotations with the help of attributes.
\begin{itemize}
\item The syntactic form $\LP\terminal{old}~\nonterminal{symbol}\RP$,
      part of $\nonterminal{relational-term}$,
      may refer to the value of global variables
      when the enclosing procedure context was entered.
      Note, in contrast to some other specification languages,
      there is no support $\LP\terminal{old}~\nonterminal{term}\RP$
      for arbitrary terms and front-end tools must push down
      the \terminal{old} to surround global variables only.
\item There are three basic assertions (B), assertions represent safety invariants (G)
      and live and non-live liveness properties (F),
      where the abbreviations mirror the corresponding LTL operators.
\item Procedure annotations (P) include pre-/postconditions.
      Modifies clauses constraint which global variables may be modified.
      Annotation \terminal{:ghost} introduces extra variables that are 
      shared between the pre- and postcondition.
      Its counterpart which takes terms as arguments is an annotation
      for call-sites that helps to avoid to introduce existential quantifiers.
\item Loop and label annotations include invariants (for safety),
      again modifies clauses,
      and decreases clauses with a given variant as an integer-valued term,
      with a possible extension for lexicographic orders
      over a list of variants.
\item Note, pre-/postconditions may make sense around arbitrary statements,
      notably loops,
      too, possibly with another syntactic form $\LP\terminal{final}~\nonterminal{symbol}\RP$,
      with refers to any variable at the end of the block.
\item Finally, we may wish to allow full LTL formulas
      but this must probably be restricted to the body statement
      of the procedure that is the entry point.
\end{itemize}

\section{Proofs and Counterexamples}

The requirement for proofs and counterexamples is so that
they can be checked by generating verification conditions in SMT-LIB.

For safety properties, a proof is given by safety annotations (G)
for all recursive procedures, loops, as well as some labels so that each cycle is broken.

For safety, a counterexample is given by a concrete execution trace
that initializes global variables, denotes the parameters
of the call to the entry procedure,
and proceeds to resolve nondeterminism in the initialization of local variables,
havoc statements, and choices.
\begin{align*}
& \LP \terminal{global}~ \nonterminal{symbol}~\nonterminal{value} \RP\rstar \\
& \LP \terminal{call}~   \nonterminal{symbol}~\nonterminal{value}\rstar \RP \\
& \LP \terminal{local}~ \nonterminal{value}\rstar \RP \\
& \LP \terminal{havoc}~ \nonterminal{value}\rstar \RP \\
& \LP \terminal{choice}~  k \RP \\
\end{align*}

For liveness, a proof is based on well-founded orders,
possibly here decreases clauses (to be clarified).

A counterexample is a lasso,
which has a concrete trace as a stem (see above).
From the entry point of the lasso,
we require a \emph{symbolic} guidance that witnesses some path
back to the entry point.
It consists of a recurrent set, annotated as an invariant,
to each loop head, including the entry point of the lasso.
All other loop heads encountered must in addition be annotated
by a decreases clause, to guarantee that the path is terminating.
On the way, havocs should be annotated with terms
(which include a choice function).
For \terminal{choice}, we may also need to resolve it symbolically,
perhaps by providing a guard attached to each possible substatement,
which encodes the condition when that path should be taken.

\section{Semantics}

\section{Tooling}

\end{document}
