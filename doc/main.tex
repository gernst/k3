\documentclass[runningheads]{llncs}
\usepackage[utf8]{inputenc}
\usepackage[T1]{fontenc}
\usepackage{graphicx} 
\usepackage{listings}
\usepackage{microtype}
\usepackage{booktabs}
\usepackage{stmaryrd}
\usepackage{mathtools}
\usepackage{amsmath,amssymb}
\usepackage{xcolor}
\usepackage{xspace}
\usepackage{tabto}
\usepackage{wrapfig}
\usepackage{comment}
\usepackage{todonotes}

\usepackage[colorlinks,allcolors=blue]{hyperref}
\usepackage[capitalise]{cleveref}

% want todonotes on margin
% \usepackage[paperwidth=155mm,
%             paperheight=235mm,
%             left=16mm,
%             right=16mm,
%             top=17mm,
%             bottom=25mm
%            ]{geometry}

% Springer-compliant names
\Crefname{equation}{Eq.}{Eqs.}
\Crefname{figure}{Fig.}{Figs.}
\Crefname{tabular}{Table}{Tabs.}
\Crefname{example}{Ex.}{Exs.}
\Crefname{section}{Sec.}{Sects.}
\Crefname{corollary}{Cor.}{Cors.}
\Crefname{proposition}{Prop.}{Props.}
\Crefname{theorem}{Thm.}{Thms.}
\Crefname{lemma}{Lem.}{Lems.}
\Crefname{algorithm}{Alg.}{Algs.}
\Crefname{definition}{Def.}{Defs.}

\definecolor{darkgreen}{rgb}{0.0, 0.53, 0.0}
\definecolor{darkblue}{rgb}{0.21,0.21,0.50}
\definecolor{darkgrey}{rgb}{0.17,0.17,0.17}

\newcommand{\code}[1]{\texttt{#1}}

\def\p#1{\mathrel{\ooalign{\hfil$\mapstochar\mkern 5mu$\hfil\cr$#1$}}}
\newcommand{\pto}{\p\to}

\newcommand{\alt}{\kern5pt\mathrel{\mid}\kern5pt}

\newcommand{\Bool}{\mathbb{B}}
\newcommand{\Nat}{\mathbb{N}}
\newcommand{\Int}{\mathbb{Z}}
\newcommand{\Real}{\mathbb{R}}

\newcommand{\Sort}[2]{#1\langle#2\rangle}

\newcommand{\Seq}[1]{\mathit{Seq}\langle#1\rangle}
\newcommand{\Set}[1]{\mathit{Set}\langle#1\rangle}
\newcommand{\MultiSet}[1]{\mathit{MultiSet}\langle#1\rangle}

\newcommand{\Map}[2]{\mathit{Map}\langle#1, #2\rangle}
\newcommand{\Array}[2]{\mathit{Array}\langle#1, #2\rangle}

\newcommand{\nonterminal}[1]{{\color{darkgreen} \langle \mathit{#1} \rangle}}
\newcommand{\terminal}[1]{{\color{darkblue} \code{\bfseries #1}}}

\newcommand{\repn}[1]{{\color{darkgrey}^{#1}}}
\newcommand{\rplus}{{\color{darkgrey}^{+}}}
\newcommand{\rstar}{{\color{darkgrey}^{*}}}
\newcommand{\ropt}{{\color{darkgrey}^{?}}}
\newcommand{\rmid}{\mathrel{\color{darkgrey}\mid}}

\newcommand{\command}[1]{{\color{darkblue} \code{\bfseries #1}}}
\newcommand{\identifier}[1]{{\color{lightblue} \code{#1}}}

\newcommand{\sem}[1]{\llbracket #1 \rrbracket}

\newcommand{\old}{\mathbf{old}}
\newcommand{\free}{\mathrm{free}}

\newcommand{\LP}{\terminal{(}}
\newcommand{\RP}{\terminal{)}}



\author{
    Gidon Ernst\inst{1}
    \and
    Alberto Griggio\inst{2}
    \and
    Martin Jonáš\inst{3}}

\institute{
    LMU Munich, Munich, Germany
    \email{gidon.ernst@lmu.de}
    \and
    Fondazione Bruno Kesseler, Trento, Italy
    \email{griggio@fbk.eu}
    \and
    Masaryk University, Brno, Czech Republic
    \email{martin.jonas@mail.muni.cz}
}

\begin{document}

\maketitle

\begin{abstract}
TODO

\keywords{Intermediate verification language}
\end{abstract}

\section{Introduction}

\clearpage

\section{Syntax}

\begin{align}
\nonterminal{statement}
    & \Coloneqq
        \LP \terminal{sequence}~
            \nonterminal{statement} \rplus
        \RP \tag{B} \\
    & \alt
        \LP \terminal{!}~
            \nonterminal{statement}~
            \nonterminal{attribute}\rplus
        \RP \tag{B} \\[4pt]
    & \alt
        \LP \terminal{assume}~
            \nonterminal{term}
        \RP \tag{AB} \\
    & \alt
        \LP \terminal{assign}~
            \LP \LP \nonterminal{symbol}~\nonterminal{term} \RP\rplus \RP
        \RP \tag{AB} \\[4pt]
    & \alt
        \LP \terminal{call}~\nonterminal{symbol}~
            \LP \nonterminal{term}\rstar \RP~
            \LP \nonterminal{symbol}\rstar \RP
        \RP \tag{P} \\
    & \alt
        \LP \terminal{return} \RP
       \tag{P$^+$} \\[4pt] 
    & \alt
        \LP \terminal{label}~
            \nonterminal{symbol}
        \RP \tag{AU} \\
    & \alt
        \LP \terminal{goto}~
            \nonterminal{symbol}
        \RP \tag{AU} \\[4pt]
    & \alt
        \LP \terminal{if}~
            \nonterminal{term}~
            \nonterminal{statement}~
            \nonterminal{statement}\ropt
        \RP \tag{S} \\
    & \alt
        \LP \terminal{while}~
            \nonterminal{term}~
            \nonterminal{statement}
        \RP \tag{S} \\
    & \alt
        \LP \terminal{break} \RP \alt \LP \terminal{continue} \RP
        \tag{S$^+$} \\[4pt]
    & \alt
        \LP \terminal{havoc}~
            \LP \nonterminal{symbol}\rplus \RP
        \RP \tag{AN} \\
    & \alt
        \LP \terminal{choice}~
            \LP \nonterminal{statement}\rplus \RP
        \RP \tag{SN} \\[8pt]
%
\nonterminal{command}
    & \Coloneqq
        \LP \terminal{declare-var}~
            \nonterminal{symbol}~\nonterminal{sort}
        \RP \notag \\
    & \alt
        \LP \terminal{define-proc}~
            \nonterminal{symbol}~
            \LP \LP \nonterminal{symbol}~\nonterminal{sort}\RP\rstar \RP \notag \\
    & \hspace*{4.42cm} \LP \LP \nonterminal{symbol}~\nonterminal{sort}\RP\rstar \RP \notag \\
    & \hspace*{4.42cm} \LP \LP \nonterminal{symbol}~\nonterminal{sort}\RP\rstar \RP
        \notag \\
    & \hspace*{3.0cm} \nonterminal{statement} \notag \RP \\
    & \alt
        \LP \terminal{annotate-tag}~
            \nonterminal{symbol}~
            \nonterminal{attribute}\rplus
        \RP \notag \\
    & \alt
        \LP \terminal{select-trace}~
            \nonterminal{trace}
        \RP \notag \\
    & \alt
        \LP \terminal{verify-call}~
            \nonterminal{symbol}~
            \LP \nonterminal{term}\rstar \RP
        \RP \notag \\
    & \alt
        \LP \nonterminal{tool-specific}\terminal{-call}~
            \nonterminal{symbol}~
            \LP \nonterminal{term}\rstar \RP
        \RP \notag \\
    & \alt
        \LP \terminal{get-proof} \RP \alt \LP \terminal{get-counterexample} \RP \notag
\end{align}

\noindent
Atomic commands (A) are those whose evaluation is to be interpreted
as taking one atomic step.
We have the following fragments, but in contrast to SMT-LIB's \terminal{set-logic},
we choose not to declare them upfront
\begin{itemize}
\item The base language (B) includes sequential composition,
      the annotation mechanism,
      as well as two atomic commands for
      parallel assignments and assumptions.
\item The interprocedural fragment (P) includes procedure calls,
      with a possible extension of return statements.
\item The fragment with unstructured control flow (U)
      contains labels and gotos.
\item The fragment with structured control flow (S)
      contains conditionals and while loops,
      with a possible extentions of breaks and continue statements.
      It makes a lot of sense to mandate that whenever
      we know a while loop resp. if statement is translated into labels and gotos,
      then the loop should also have a \terminal{:named} annotation
      that becomes the label of the loop head.
\item The nondeterministic fragment (N)
      includes atomic nondeterministic assignments,
      as well as structured choice.
\item The concurrent fragment (C) is yet to be defined
      and left as an exercise to the reader, respectively,
      for future work.
\item Similarly, we aim to standardize exceptions (E) and a memory interface for the heap (H).
\item It is debatable whether there should be a specification statement
      which introduces a nondeterministic transition as a combination
      of \terminal{havoc} and \terminal{assume}.
      It possibly avoids introduction of additional local variables,
      but introduces \terminal{old} into the expression language of programs,
      which would otherwise only occur in specifications.
\end{itemize}

\noindent
The command language for all fragments encompasses
\begin{itemize}
\item Declarations of global variables,
      which are implicitly in scope in all subsequent procedure declarations.
\item Named procedure declarations,
      with lists of inputs, outputs, and local variables,
      followed by a single body statement.
\item The annotation command permits to link attributes
      to statements with a given tag.
\item The command \terminal{select-trace} specifies an execution trace
      to the violation of a safety counterexample, respectively,
      to the loop head of a lasso for counterexamples to liveness.
      Maybe it is not entirely necessary to include this into the top-level commands,
      however, it will be part of the responses by the solvers,
      just like \terminal{annotate-tag}.
\item The \terminal{verify-call} command takes a similar role as the \terminal{check-sat} command,
      where the given symbol is a procedure name to enter,
      and the list of terms for inputs,
      which may refer to globally declared constants and functions,
      and which may be constrained by prior \terminal{assert} commands.

      The command returns either \terminal{correct} (all specified properties hold),
      \terminal{incorrect} (at least one property is violated),
      or \terminal{unkown} (inconclusive, e.g. due to timeout).
\item Analogously to SMT-LIB, the commands \terminal{get-proof}
      and \terminal{get-counterexample} produce evidence
      for the most-recent result of \terminal{verify-call}.
\item Extentsions on queries supported by tools may introduce additional commands.
\end{itemize}

\section{Properties}

\begin{align}
& \terminal{:assert}
    && \nonterminal{relational-term}
    \tag{GB} \\
& \terminal{:live}
    \tag{FB} \\
& \terminal{:not-live}
    \tag{FB} \\[4pt]
& \terminal{:ghost}
    && \LP \nonterminal{symbol}\rplus \RP
    \tag{P} \\
& \terminal{:ghost}
    && \LP \nonterminal{term}\rplus \RP
    \tag{P} \\
& \terminal{:requires}
    && \nonterminal{term}
    \tag{GP} \\
& \terminal{:ensures}
    && \nonterminal{reltional-term}
    \tag{GP} \\
& \terminal{:invariant}
    && \nonterminal{relational-term}
    \tag{GS} \\
& \terminal{:decreases}
    && \terminal{term}
    \tag{FS} \\
& \terminal{:decreases}
    && \LP \terminal{term}\rplus \RP
    \tag{FS$^+$} \\
& \terminal{:modifies}
    && \LP \nonterminal{symbol}\rplus \RP
    \tag{GPS} \\[4pt]
& \terminal{:ltl}
    && \nonterminal{ltl-formula}
    \tag{LTL}
\end{align}

\noindent
Properties are specified as annotations with the help of attributes.
\begin{itemize}
\item The syntactic form $\LP\terminal{old}~\nonterminal{symbol}\RP$,
      part of $\nonterminal{relational-term}$,
      may refer to the value of global variables
      when the enclosing procedure context was entered.
      Note, in contrast to some other specification languages,
      there is no support $\LP\terminal{old}~\nonterminal{term}\RP$
      for arbitrary terms and front-end tools must push down
      the \terminal{old} to surround global variables only.
\item There are three basic assertions (B), assertions represent safety invariants (G)
      and live and not-live liveness properties (F),
      where the abbreviations mirror the corresponding LTL operators.
\item Procedure annotations (P) include pre-/postconditions.
      Modifies clauses constraint which global variables may be modified.
      Annotation \terminal{:ghost} introduces extra variables that are 
      shared between the pre- and postcondition.
      Its counterpart which takes terms as arguments is an annotation
      for call-sites that helps to avoid to introduce existential quantifiers.
\item Loop and label annotations include invariants (for safety),
      again modifies clauses,
      and decreases clauses with a given variant as an integer-valued term,
      with a possible extension for lexicographic orders
      over a list of variants.
\item Note, pre-/postconditions may make sense around arbitrary statements,
      notably loops,
      too, possibly with another syntactic form $\LP\terminal{final}~\nonterminal{symbol}\RP$,
      with refers to any variable at the end of the block.
\item Finally, we may wish to allow full LTL formulas
      but this must probably be restricted to the body statement
      of the procedure that is the entry point.
\end{itemize}

\section{Proofs and Counterexamples}

We outline the interaction between the commands and the solver.
Just like in SMT-LIB \terminal{verify-call} command is responded
to by one of the following status flags:
\terminal{correct}, \terminal{incorrect}, and \terminal{unknown}.
The requirement for proofs and counterexamples is so that
they can be checked by generating verification conditions in SMT-LIB.

\subsection{Safety}

For safety properties, a proof is given by safety annotations (G)
for all recursive procedures, loops, as well as some labels so that each cycle is broken.
We describe the interaction from the point of the client of the verifier,
where $\RES$ is a response of the verifier
and $\QRY$ is a follow-up query from the client.
\begin{align*}
\QRY~ & \LP \terminal{verify-call}~
            \nonterminal{symbol}~
            \LP \nonterminal{term}\rstar \RP
        \RP \notag \\
\RES~ & \terminal{correct} \\[4pt]
\QRY~ & \LP \terminal{get-proof} \RP \\
\RES~ & \LP \terminal{annotate-tag}~
           \nonterminal{symbol}~
           \nonterminal{attribute}\rplus
                 \RP \rstar
\end{align*}
A counterexample to safety (G) is given by a model for the SMT-LIB
definitions as well as a concrete execution trace:
\begin{align*}
\QRY~ & \LP \terminal{verify-call}~
            \nonterminal{symbol}~
            \LP \nonterminal{term}\rstar \RP
        \RP \notag \\
\RES~ & \terminal{incorrect} \\[4pt]
\QRY~ & \LP \terminal{get-counterexample} \RP \\
\RES~ & \LP \terminal{model}~  \nonterminal{model} \RP \\
\RES~ & \LP \terminal{select-trace}~  \nonterminal{trace} \RP
\end{align*}
The trace initializes global variables, denotes the parameters
of the call to the entry procedure,
and proceeds to resolve nondeterminism in the initialization of local variables,
havoc statements, and choices.
\begin{align*}
\nonterminal{trace}
    & \Coloneqq
            \LP \terminal{global}~ \nonterminal{symbol}~\nonterminal{value} \RP\rstar \\
    & \altspace \LP \terminal{call}~   \nonterminal{symbol}~\nonterminal{value}\rstar \RP \\
    & \altspace     \nonterminal{step}\rstar \\
    & \altspace \LP \terminal{incorrect-tag}~\nonterminal{symbol}~
                        \nonterminal{attribute}\rplus \RP \\[4pt]
\nonterminal{step}
    & \Coloneqq \LP \terminal{local}~  \LP \nonterminal{value}\rstar \RP \RP \\
    & \alt      \LP \terminal{havoc}~  \LP \nonterminal{value}\rstar \RP \RP \\
    & \alt      \LP \terminal{choice}~  k \RP \RP\rstar
\end{align*}
The trace ends with an indication which property exactly was violated,
where the symbol coincides with some tag representing
the annotation that is incorrect and the attributes
are a subset of those that are attached to that tag.

\subsection{Liveness}

For liveness, a proof is based on well-founded orders,
possibly here decreases clauses (to be clarified).
Liveness proofs use exactly the same format as safety proofs,
albeit their validation of course takes live and not-live locations into account
(see below).

A counterexample is a lasso,
which has a concrete trace as a stem, just as for counterexamples to safety,
that navigates to the head of a loop.
The \terminal{incorrect-tag} at the end of the trace
identifies the head of the head of a loop that has some infinite path,
which is encoded by a recurrent set via the \terminal{:invariant} tag,
but the validator is using underapproximation.
\begin{align*}
\QRY~ & \LP \terminal{verify-call}~
            \nonterminal{symbol}~
            \LP \nonterminal{term}\rstar \RP
        \RP \notag \\
\RES~ & \terminal{incorrect} \\[4pt]
\QRY~ & \LP \terminal{get-counterexample} \RP \\
\RES~ & \LP \terminal{model}~  \nonterminal{model} \RP \\
\RES~ & \LP \terminal{select-trace}~  \nonterminal{trace} \RP \\
\RES~ & \LP \terminal{annotate-tag}~
           \nonterminal{symbol}~
           \nonterminal{attribute}\rplus
                 \RP \rstar
\end{align*}

% For \terminal{choice}, we may also need to resolve it symbolically,
% perhaps by providing a guard attached to each possible substatement,
% which encodes the condition when that path should be taken.
% 
% Moreover, it seems very beneficial to include branching information,
% so that a validator does not need to fully evaluate guard expressions,
% which may be tricky if they contain non-executable stuff, such as quantifiers.

\section{Validation}

The main design goal of proof and counterexample formats is
that they can be compiled together with the original program
into an SMT-LIB file.

\todo[inline]{
    Interlude: The stuff below is almost right.
    The trick is to proceed by defining predicate transformers to keep track of live and not-live locations encountered so far.}

We define $[\alpha]^{L,N}~\phi$ to track a set~$L$ of live locations
that are still \emph{expected to be encountered}, starting with~$L$
as all live locations occurring inside the loop body,
and~$N$ is the set of not-live locations that \emph{were actually encountered},
starting with $N = \varnothing$.

A liveness proof must then show that some measure~$\delta$
decreases through the loop, but only if either~$L$ is not empty
(a live location was missed),
or~$N$ is not empty
(a not-live location was hit),
both of which should not happen infinitely often.

The scheme must however be updated whenever we have groups of live and not-live locations,
in particular when it is sufficient to visit \emph{one} location of a live group.
This prompts a switch to compact VCs like in a collecting semantics.

\todo[inline]{End interlude.}

Validation relies on both standard over-approximate reasoning
as well as underapproximate reasoning that denotes the existence of a path.
We use notation from Dynamic Logic to represent these as follows:
\begin{align*}
\nonterminal{formula}
    & \Coloneqq \BOX{\nonterminal{statement}}{\nonterminal{property}} \\
    & \alt      \DIA{\nonterminal{statement}}{\nonterminal{property}} \\
    & \alt \dots \\[4pt]
\nonterminal{property}
    & \Coloneqq \LP \terminal{assert}~ \nonterminal{symbol}~ \phi \RP \\
    & \alt      \LP \terminal{live}~ \nonterminal{symbol}\rplus \RP \\
    & \alt      \LP \terminal{not-live}~ \nonterminal{symbol}\rplus \RP
\end{align*}
where the symbols in the properties point to tags
and the semantics of modalities
$\BOX{\alpha}{P}$ and
$\DIA{\alpha}{P}$ are defined
for statements~$\alpha$ and properties~$P$ as follows:
\begin{align*}
s \models \BOX{\alpha}{P}
    & \iff
        \forall\ \sigma.\ \sigma(0) = s \land \sigma \models \alpha
                \implies \sigma \models P \\[4pt]
s \models \DIA{\alpha}{P}
    & \iff
        \exists\ \sigma.\ \sigma(0) = s \land \sigma \models \alpha
                \land \sigma \models P
\end{align*}
so that these two modalities are dual to each other.

To make this work, we assume the semantics of finite execution of statements
is extended with a stuttering loop that is implicitly annotated by
\terminal{:tag} \code{exit}.
This ensures that all executions are in fact infinite.
    \todo{document this!}

Modality $\BOX{\alpha}{P}$ resembles Dijkstra's weakest liberal precondition.
As a predicate transformer, it unfolds the program as follows:
    \todo{add unfolding of safety and liveness conditions in atomic commands}
\begin{align*}
\BOX{\LP \terminal{sequence}~ \alpha_1, \dots, \alpha_n \RP}{P}
    &\quad \equiv \quad
         \BOX{\alpha_1}{} \cdots {\BOX{\alpha_n}{P}}
    \\
\BOX{\LP \terminal{assume}~\phi \RP}{P}
    &\quad \equiv \quad
         \phi \implies P
    \\
\BOX{\LP \terminal{assign}~\LP \LP x_1~e_1 \RP \dots \LP x_n~e_n \RP \RP}{P}
    &\quad \equiv \quad
         \forall\ x'_1 = e_1, \dots, x'_n = e_n.\ P'
    \\
\BOX{\LP \terminal{call}~ \rho~ \LP e_1 \dots e_n \RP~
                               \LP x_1 \dots x_m \RP \RP}{P}
    &\quad \equiv \quad
        \text{(apply contract of $\rho$)}
    \\
\BOX{\LP \terminal{if}~\phi~\alpha~\beta \RP}{P}
    &\quad \equiv \quad
        \LP \terminal{ite}~\phi~
                \BOX{\alpha}{P}~
                \BOX{\beta}{P} \RP
    \\
\BOX{\LP \terminal{while}~\phi~\alpha~\beta \RP}{P}
    &\quad \equiv \quad
        \text{(summarize loop)}
    \\
\BOX{\LP \terminal{havoc}~\LP x_1 \dots x_n \RP \RP}{P}
    &\quad \equiv \quad
         \forall\ x'_1, \dots, x'_n.\ P'
    \\
\BOX{\LP \terminal{choice}~\LP \alpha_1 \dots \alpha_n \RP \RP}{P}
    &\quad \equiv \quad
         \BOX{\alpha_1}{P} \land \dots \land \BOX{\alpha_n}{P}
\end{align*}
Modality $\BOX{\alpha}{P}$ is dual:
\begin{align*}
\DIA{\LP \terminal{sequence}~ \alpha_1, \dots, \alpha_n \RP}{P}
    &\quad \equiv \quad
         \DIA{\alpha_1}{} \cdots {\DIA{\alpha_n}{P}}
    \\
\DIA{\LP \terminal{assume}~\phi \RP}{P}
    &\quad \equiv \quad
         \phi \land P
    \\
\DIA{\LP \terminal{assign}~\LP \LP x_1~e_1 \RP \dots \LP x_n~e_n \RP \RP}{P}
    &\quad \equiv \quad
         \exists\ x'_1 = e_1, \dots, x'_n = e_n.\ P'
    \\
\DIA{\LP \terminal{call}~ \rho~ \LP e_1 \dots e_n \RP~
                               \LP x_1 \dots x_m \RP \RP}{P}
    &\quad \equiv \quad
        \text{(apply contract of $\rho$)}
    \\
\DIA{\LP \terminal{if}~\phi~\alpha~\beta \RP}{P}
    &\quad \equiv \quad
        \LP \terminal{ite}~\phi~
                \DIA{\alpha}{P}~
                \DIA{\beta}{P} \RP
    \\
\DIA{\LP \terminal{while}~\phi~\alpha~\beta \RP}{P}
    &\quad \equiv \quad
        \text{(summarize loop)}
    \\
\DIA{\LP \terminal{havoc}~\LP x_1 \dots x_n \RP \RP}{P}
    &\quad \equiv \quad
         \exists\ x'_1, \dots, x'_n.\ P'
    \\
\DIA{\LP \terminal{choice}~\LP \alpha_1 \dots \alpha_n \RP \RP}{P}
    &\quad \equiv \quad
         \DIA{\alpha_1}{P} \lor \dots \lor \DIA{\alpha_n}{P}
\end{align*}

\section{Outtakes / Design Alternatives}

Whether to include branching information in counterexample traces:
Given a model, we can resolve the latter two branchings by \emph{concrete} execution.
However, if the branchings are included, the counterexample trace
can be followed syntactically, in which case we can also validate the counterexample
by \emph{concolic} execution, so that the SMT solver does not necessarily need to rely on the given model at all but will reconstruct (some) that makes the trace feasible.
However, one of the two is strictly required.
\begin{align*}
& \LP \terminal{if}~     \nonterminal{value} \RP\ropt \\
& \LP \terminal{while}~  \nonterminal{value} \RP\ropt \\
\end{align*}

\end{document}
